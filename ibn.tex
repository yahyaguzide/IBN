%	lecture: IBN( Introduction to OS's and Networking )
%	date(dd/mm/yyyy): 19/04/2019
%	lecture is held by Prof. Dr. Artur Adrzejak
%	
%	Yahya Guezide

\documentclass[a4paper, 11pt, twoside]{article}

\author{Prof. Dr. Artur Andrzejak}
\title{Einf\"uhrung in Betriebssysteme und Netzwerke}
\date{\today}

% to write Roman Numerals use \rom{num}
\makeatletter
\newcommand*{\rom}[1]{\expandafter\@slowromancap\romannumeral #1@}
\makeatother

\usepackage[document]{ragged2e}
\begin{document}
	\maketitle
	\newpage

	\tableofcontents
	\newpage

	\section{Was ist ein Betriebssystem}
	Betriebsysteme sind \"uberall, in unseren PC's, Laptops, Handy's
	ja sogar Milchpump-maschinen ben\"otigen ein Betriebssystem.
	\linebreak
	Nun was ist ein Betriebssystem?
	\linebreak

	vereinfacht \textbf{{\em``Ein Program, das immer laufen muss''}} ??
	\linebreak\linebreak
	Eine richtige Definition ist dies nicht, ein besserer ansatz w\"are
	\linebreak
	\textbf{ {\em``Die Softwareschicht zwischen Hardware und Anwendungsprogrammen''}}
	\linebreak
	nehmen wir f\"ur denn moment diese Definition so an, was genau ein Betriebssystem
	ist und was es alles macht werden wir im verlauf lernen und besser verstehen.

	\subsection{Komplexität von Betriebssystemen}
	Betrachten wir ein paar der bekanntesten Systeme
	\linebreak

	Gebebene Werte sind in Millionen
	% ToDo: Quelle angeben
	\begin{itemize}
		\item OpenSolaris			9.7
		oitem Linux Kernel2.6.32	12.6
		\item Mac Os X 10.4			84
		\item Debian 4.0/5.0		283/324 
	\end{itemize}
%	\linebreak

	Auch Windows hat sich drastisch ver\"andert und dem Zeitgeist angepasst
	% ToDo: liste etwas weiter nach rechts verschieben 
	\begin{description}
		\item[1993] Windows NT 3.1	4-5
		\item[1994] Windows NT 3.5	7-8
		\item[1996] Windwos NT 4.0	11-12
		\item[2000] Windows 2000		>29
		\item[2001] Windows XP		40
		\item[2006] Windows Vista	50
		\item[2009] Windows 7		40
		\item[2010] Windows 10		27-50
	\end{description}

	\subsection{Rechner Erster Generation}
	\label{sec:rechnererstergeneration}
	\begin{description}
		\item[1941] Z3 von Konrad Zuse, Berlin (Relais)
		\item[1943] Colossus in Bletchley Park, UK (2500 R\"ohren)
		\item[1944] Mark\rom{1} in Harvard Univ. (Relais, Schalter)
		\item[1946] Eniac von William Mauchley / J. Presper Eckert Univ. Pennsylvania (17.468 Elek R\"ohren)
	\end{description}
	
	Rechner der Ersten Generation bestanden aus Relais und Elektronenr\"ohren, bevor es spezelle sprachen
	zum bedienen der rechner gab wurden programme durch ``umstecken'' von Kabeln
	und oder mit hilfe von lochkarten eingelesen.
	\linebreak
	Die Elektronenr\"ohren waren sehr empfindlich und teuer dies bedeutete
	das jeder fehler im programm zu langen wartezeiten und zu einem anstieg der kosten f\"uhrte.
	Diese Rechner hatten noch kein Betriebssystem, die Hardware wurde direkt verwendet.
	\linebreak\linebreak
	Als hersteller anfingen spezielle sprachen f\"ur die rechner zu entwickeln setzte sich Assembler durch
	und die Programme wurden f\"ur diese Rechner mit Assambler und sp\"ater mit Fortran geschrieben.

	% ToDO: add video links from vl01-ibn.pdf at page 7
	
	\subsubsection{Die Von Neuman-Architektur (VNA)}
	Der Eniac[\ref{sec:rechnererstergeneration}] hatte keinen Programmspeicher, die programmierung erfolgte per
	umstecken verschiedener Kabel.
	\linebreak

	John von Neuman ve\"offentlichte 1945 das Konzept der VNA(Princton Architektur),
	dieses ist ein Schaltungskonzept  f\"ur einen Universellen Rechner.
	In diesem Konzept enthielt der Speicher Daten und Programcode.
	Eine der fr\"uhesten Rechner der diese Architektur nutze war der Evac,
	der postdecessor von Eniac[\ref{sec:rechnererstergeneration}] welcher 


\end{document}
